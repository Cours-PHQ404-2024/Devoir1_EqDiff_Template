%! Author = gince
%! Date = 1/18/2024

% Preamble
\documentclass[11pt]{article}

% Packages
\usepackage{amsmath, amssymb}
\usepackage{physics}
\usepackage[french]{babel}
\usepackage[margin=2.5cm]{geometry}
\usepackage[utf8]{inputenc}
\usepackage[T1]{fontenc}
\usepackage{hyperref}
\usepackage{caption}
% Document
\begin{document}

\section*{Modélisation du Mouvement d'une Particule dans un Plan sous l'Influence d'une Force de Linéaire}
\label{sec:modele}


\noindent Considérons le mouvement d'une particule de masse \(m\) dans un espace en deux dimensions sous l'influence
d'une force de rappel linéaire.

\bigskip

\noindent La force appliquée sur la particule est donnée par:

\[
\mathbf{f}(\mathbf{r}) = -k \mathbf{r}
\]

\noindent où \(\mathbf{r} = (x, y)\) est le vecteur position de la particule et \(k\) est la constante de raideur.

\bigskip

\noindent La force de rappel linéaire est une force restauratrice qui est proportionnelle à la position de la particule et
dirigée vers l'origine.
En d'autres termes, plus la particule s'éloigne de l'origine, plus la force de rappel est forte,
et elle est dirigée vers l'origine.

\bigskip

\noindent La deuxième loi de Newton pour cette force est formulée comme une équation vectorielle du second ordre:

\[
m \frac{d^2\mathbf{r}}{dt^2} = \mathbf{f}(\mathbf{r})
\]

\noindent En supposant une masse unitaire (\(m = 1\)), cette équation se simplifie à:

\[
\frac{d^2\mathbf{r}}{dt^2} = -k \mathbf{r}
\]

\noindent Cette équation vectorielle peut être décomposée en deux équations différentielles couplées, une pour chaque
composante de \(\mathbf{r} = (x, y)\).
En effet,

\begin{align*}
    \mathbf{r} &= \mqty(x \\ y) \\
    \frac{d\mathbf{r}}{dt} &= \mqty(\frac{dx}{dt} \\ \frac{dy}{dt}) \\
    \frac{d^2\mathbf{r}}{dt^2} &= \mqty(\frac{d^2x}{dt^2} \\ \frac{d^2y}{dt^2})
\end{align*}

\noindent et puisque

\begin{align*}
    \mathbf{f}(\mathbf{r}) &= -k \mqty(x \\ y) \\
    \frac{d\mathbf{f}}{dt} &= -k\mqty(\frac{dx}{dt} \\ \frac{dy}{dt}) \\
    \frac{d^2\mathbf{f}}{dt^2} &= -k\mqty(\frac{d^2x}{dt^2} \\ \frac{d^2y}{dt^2})
\end{align*}

\noindent donc plus explicitement, on obtient:

\[
\begin{align*}
\frac{d^2x}{dt^2} &= -k x, \\
\frac{d^2y}{dt^2} &= -k y.
\end{align*}
\]

\noindent Ces équations décrivent comment les positions \(x\) et \(y\) de la particule évoluent au fil du temps sous
l'influence de la force de rappel linéaire.

\bigskip

\noindent Maintenant, pour décrire l'évolution du système, on peut utiliser un vecteur
d'état \(\mathbf{v} = [x, y, v_x, v_y]\), où \(v_x\) et \(v_y\) sont les composantes de la vitesse dans les
directions \(x\) et \(y\) respectivement.
Les dérivées de ce vecteur d'état par rapport au temps sont données par:

\[
\begin{align*}
\frac{dx}{dt} &= v_x, \\
\frac{dv_x}{dt} &= -k x, \\
\\
\frac{dy}{dt} &= v_y, \\
\frac{dv_y}{dt} &= -k y.
\end{align*}
\]

\noindent Ces équations décrivent comment les composantes du vecteur d'état évoluent au fil du temps, fournissant une
description complète du mouvement de la particule sous l'influence de la force de rappel linéaire.
Finalement, vous pouvez remarquer quant utilisant le vecteur d'état, on a réduit le système d'équations différentielles
d'ordre deux à un système d'équations différentielles d'ordre un.




\end{document}